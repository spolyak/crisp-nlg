\section{Deep and shallow semantic construction}
\label{sec:motivation}

We start with an informal illustration of why \rmrs\ is necessary and
useful to represent partial semantic information.  Consider the
following (toy) sentence:

\begin{examples}
  \item Every fat cat chased some dog.
\end{examples}

This sentence exhibits several kinds of ambiguity, including a scope
ambiguity between ``every fat cat'' and ``some dog'' and lexical
ambiguities of the nouns ``cat'' and ``dog'' (which have 8 and 7
WordNet senses respectively).  Simplifying slightly by ignoring tense
information, two of these readings are shown as logical forms below;
these can be represented as trees as shown in Fig.~\ref{fig:1}.
%  (but see Section~\ref{sec:extensions})

\begin{examples}
\item $\sem{\_every\_q\_1}(x, \sem{\_fat\_j\_1}(e',x) \wedge
    \sem{\_cat\_n\_1}(x),$\\
\hspace*{0.1in} $\sem{\_some\_q\_1}(y, \sem{\_dog\_n\_1}(y),$\\
\hspace*{0.2in}$\sem{\_chase\_v\_1}(e,x,y)))$
\label{ex:fat-cat-1}
\item $\sem{\_some\_q\_1}(y, \sem{\_dog\_n\_2}(y),$\\
\hspace*{0.1in}$\sem{\_every\_q\_1}(x, \sem{\_fat\_j\_1}(e',x) \wedge
    \sem{\_cat\_n\_2}(x), $\\
\hspace*{0.2in}$\sem{\_chase\_v\_1}(e,x,y)))$
\label{ex:fat-cat-2}
\end{examples}


\begin{figure}[t]
\centering
\includegraphics[scale=0.8]{pic-cat-chased-dog} \\
\includegraphics[scale=0.8]{pic-cat-chased-dog-2}
\caption{Structure trees of the semantic representations (\ref{ex:fat-cat-1}) and
  (\ref{ex:fat-cat-2}). \label{fig:1}}
\end{figure}


Now imagine we are trying to extract semantic information from the
output of a part-of-speech tagger by using the word lemmas as lexical
predicate symbols.  Such a semantic representation is highly partial,
as it will not say anything about predicate-argument relations or
resolve lexical ambiguity---it will use predicate symbols such as
$\sempred{\_cat\_n}$, which might resolve to the predicate symbols
$\sem{\_cat\_n\_1}$ or $\sem{\_cat\_n\_2}$ in the complete semantic
representation.  (Notice the different fonts for the ambiguous and
unambiguous predicate symbols.)

Partial semantic information is
typically represented using an underspecification formalism such as
\mrs\ \cite{copestake:etal:2005} or {\sc clls} \cite{egg:etal:inpress}.  But
most underspecification formalisms are unable to represent semantic
information that is as partial as what we get from a {\sc pos} tagger: They
assume that the complete predicate-argument structure is known.
\rmrs\ \cite{copestake:2007a} was
designed to address this problem.  In \rmrs, we can express the
information we get from the {\sc pos} tagger as follows:

\begin{examples}
\item \label{ex:cat-pos}
$l_1:a_1:\sempred{\_every\_q}(x_1)$, \\
$l_{41}:a_{41}:\sempred{\_fat\_j}(e')$,\\
$l_{42}:a_{42}:\sempred{\_cat\_n}(x_3)$\\
$l_5:a_5:\sempred{\_chase\_v}(e)$, \\
$l_6:a_6:\sempred{\_some\_q}(x_6)$, \\
$l_9:a_9:\sempred{\_dog\_n}(x_7)$
\end{examples}

This \rmrs\ expresses only that certain predications must be made in
the semantic representation -- but it doesn't say anything about the
structure of the semantic representation, about most arguments of the
predicates (note that $\sempred{\_chase\_v}(e)$ doesn't say who chases
whom), or about the coindexation of variables ($\sempred{\_every\_q}$
binds the variable $x_1$, whereas $\sempred{\_cat\_n}$ speaks about
$x_3$), and maintains the lexical ambiguities.  Technically, it
consists of six \emph{elementary predications} ({\sc ep}s), one for
each word lemma in the sentence; each of them is prefixed by a
\emph{label} and an \emph{anchor}, which are essentially variables
that refer to nodes in the trees in Fig.~\ref{fig:1}.  We can say that
the two trees \emph{satisfy} this \rmrs\ because it is possible to map
the labels and anchors into nodes in each tree and variable names like
$x_1$ and $x_3$ into variable names in the tree in such a way that the
predication of the node that labels and anchors are mapped to is
consistent with that given in the {\sc ep}s of the {\sc rmrs}.  For
instance, the first tree satisfies the \rmrs\ because $l_1$ and $a_1$
can map to the root of that tree and $x_1$ to $x$, and the root label
$\sem{\_every\_q\_1}$ is consistent with the {\sc ep} predicate
$\sempred{\_every\_q}$.

% The lack of semantic dependencies rendered by a {\sc pos} tagger means
% we have unique arguments to each predicate symbol (e.g., see $x_1$ and
% $x_3$ in (\ref{ex:cat-pos}), compared with the {\em same} variable $x$
% in the semantic representations (\ref{ex:fat-cat-1}) and
% (\ref{ex:fat-cat-2})).

There are of course many other trees (and thus, fully specific
semantic representations such as (\ref{ex:fat-cat-1})) that are
described equally well by the \rmrs\ (\ref{ex:cat-pos}); this is not
surprising, given that the semantic information we got from the {\sc
  pos} tagger is so incomplete.  If we have more information, say
information about subject and object relations from a chunk parser
like Cass \cite{abney:1996}, we can represent these in a more detailed
\rmrs, as follows:

\begin{examples}
\item 
$l_1:a_1:\sempred{\_every\_q}(x_1)$, \\
$l_{41}:a_{41}:\sempred{\_fat\_j}(e')$,\\
$l_{42}:a_{42}:\sempred{\_cat\_n}(x_3)$\\
$l_5:a_5:\sempred{\_chase\_v}(e)$, \\
\hspace*{0.1in} $\ARG_1(a_5,x_4),
\ARG_2(a_5,x_5)$\\ 
$l_6:a_6:\sempred{\_some\_q}(x_6)$, \\
$l_9:a_9:\sempred{\_dog\_n}(x_7)$\\
$x_3=x_4$, $x_5=x_7$
\label{ex:cat-partial-parser}
\end{examples}

This \rmrs\ uses two new types of atoms.  Atoms of the form $x_3=x_4$
express that the two variables $x_3$ and $x_4$ must map to the same
variable in any fully specific logical form; e.g., both to the
variable $x$ in Fig.~\ref{fig:1}.  Atoms of the form $\ARG_i(a,z)$ and
$\ARG_i(a,h)$ express that the $i$-th child (counting from 0) of the
node to which the anchor $a$ refers is the variable name that $z$
denotes (or the node that the {\em hole} $h$ denotes).  So unlike
earlier underspecification formalisms, \rmrs\ can specify the
predicate of an atom separately from its (scopal and non-scopal)
arguments; this is necessary for supporting parsers where information
about lexical subcategorisation is absent. If we also allow atoms of
the form $\ARG_{\{2,3\}}(a,x)$ to express uncertainty as to whether
$x$ is the second or third child of the anchor $a$, then \rmrs\ can
even specify the arguments to a predicate while underspecifying their
position.  This is useful for specifying arguments to \_give\_v when a
parser that is unableto process unbounded dependencies is faced with
{\em Which bone did you give the dog?} vs.\ {\em To which dog did you
  give the bone?}

Finally, the \rmrs\ (\ref{ex:cat-erg}) is a notational variant of the
{\sc mrs} derived by the {\sc erg}, a wide-coverage deep grammar:
\begin{examples}
\item $l_1:a_1\handel\sempred{\_every\_q\_1}(x_1),$\\
\hspace*{0.1in}$\mathsf{RSTR}(a_1,h_2),
\mathsf{BODY}(a_1,h_3)$\\ 
$l_{41}:a_{41}\handel\sempred{\_fat\_j\_1}(e'), \ARG_1(a_{41},x_2)$\\
$l_{42}:a_{42}\handel\sempred{\_cat\_n\_1}(x_3)$\\
$l_5:a_5\handel\sempred{\_chase\_v\_1}(e)$,\\
\hspace*{0.1in}$\ARG_1(a_5,x_4),
\ARG_2(a_5,x_5)$\\ 
$l_6:a_6\handel\sempred{\_some\_q\_1}(x_6)$,\\
\hspace*{0.1in}$\mathsf{RSTR}(a_6,h_7),
\mathsf{BODY}(a_6,h_8)$\\ 
$l_9:a_9\handel\sempred{\_dog\_n\_1}(x_7)$\\
$h_2=_q l_{42}, l_{41}=l_{42}, h_7 =_q l_9$\\
$x_1=x_2, x_2=x_3, x_3=x_4,$\\
$x_5=x_6, x_5=x_7$
\label{ex:cat-erg}
\end{examples}

Here we have full information about the predicate-argument structure
and the variable binding; we only underspecify scope and some lexical
ambiguity.  $\mathsf{RSTR}$ and $\mathsf{BODY}$ are conventional names
for the $\ARG_1$ and $\ARG_2$ of a quantifier predicate symbol.  Atoms
like $h_2 \qeq l_{42}$ (``qeq'') specify a certain kind of
``outscopes'' relationship between the hole and the label, and are
used here to underspecify the scope of the two quantifiers.

Notice that the labels of the {\sc ep}s for ``fat'' and ``cat'' are
stipulated to be equal in this \rmrs, whereas the anchors are not.  In
the tree, it is the anchors that are mapped to the nodes with the
labels $\sem{\_fat\_j\_1}$ and $\sem{\_cat\_n\_1}$; the label is
mapped to the conjunction node just above them.  In other words, the
role of the anchor in an {\sc ep} is to connect a predicate to its
arguments, while the role of the label is to connect the {\sc ep} to
the surrounding formula.  Representing conjunction with label sharing
stems from \mrs\ and provides compact semantic representations.

Finally, (\ref{ex:cat-erg}) uses predicate symbols like
$\sempred{\_dog\_n\_1}$ that are meant to be more specific than the
symbols like $\sempred{\_dog\_n}$ which the earlier \rmrs s used.
This reflects the fact that the deep grammar will perform some lexical
disambiguation, whereas the chunker and {\sc pos} tagger didn't.  The fact
that the former symbol should be more specific than the latter can be
represented using SPEC atoms like $\sempred{\_dog\_n\_1}\sqsubseteq
\sempred{\_dog\_n}$), which we have omitted in (\ref{ex:cat-erg}) for
brevity.  Note that even a deep grammar will not fully disambiguate to
semantic predicate symbols, such as WordNet senses, and so
$\sempred{\_dog\_n\_1}$ can still be consistent with multiple symbols
like $\sem{\_dog\_n\_1}$ and $\sem{\_dog\_n\_2}$ in the semantic
representation.

In summary, \rmrs\ is a formalism for partial semantic
representations.  It allows us to represent the semantic information
that can be extracted from a wide range of {\sc nlp} tools in a uniform way.
This is useful for hybrid systems which fall back to shallower
analyses if deeper parsing is unsuccessful, or which try to match
deeply parsed queries against shallow parses of large text corpora;
and in fact, \rmrs\ is gaining popularity as a practical interchange
format for exactly these purposes \cite{copestake:2003,frank:2004}.
However, \rmrs\ is still relatively ad-hoc in that its formal syntax
is not consistent across different papers, and its formal semantics is
not defined; we don't know, formally, what an \rmrs\ \emph{means} in
terms of semantic representations like (\ref{ex:fat-cat-1}) and
(\ref{ex:fat-cat-2}), and this hinders our ability to design efficient
algorithms for processing \rmrs. The purpose of this paper is to lay
the groundwork for fixing this problem.

%%% Local Variables: 
%%% mode: latex
%%% TeX-master: "rmrs-08"
%%% End: 
