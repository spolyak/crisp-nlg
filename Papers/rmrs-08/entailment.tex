\section{Future work}
\label{sec:entailment}

The above definitions serve an important theoretical purpose in that
they provide a formal underpinning for the use of \rmrs\ in practical
systems.  Next to the peace of mind that comes with the use of a
well-understood formalism, we hope that the work reported here will
serve as a starting point for future research.

One direction which we believe may develop from this paper is the
development of efficient solvers for \rmrs.  As a first step, it would
be interesting to define a practically useful fragment of \rmrs\ with
polynomial-time satisfiability.  Our definition is sufficiently close
to that of dominance constraints that we expect that it should be
feasible to carry over the definition of \emph{normal dominance
  constraints} \cite{Althaus_etal:JoA} to \rmrs; neither the lexical
ambiguity of the node labels nor the separate specification of
predicates and arguments should make satisfiability harder.

Furthermore, the above definition of \rmrs\ provides new concepts
which can help us phrase questions of practical grammar engineering in
well-defined formal terms.  For instance, one crucial issue in
developing a hybrid system that combines or compares the outputs of
deep and shallow processors is to determine whether the \rmrs s
produced by the two systems are compatible.  In the new formal terms,
we can characterise compatibility of a more detailed \rmrs\ $\varphi$
(perhaps from a deep grammar) and a less detailed \rmrs\ $\varphi'$
simply as \emph{entailment} $\varphi \models \varphi'$.  If entailment
holds, this tells us that all claims that $\varphi'$ makes about the
semantic content of a sentence are consistent with the claims that
$\varphi$ makes.

At this point, we cannot provide an efficient algorithm for testing
entailment of \rmrs.  However, we propose the following novel
syntactic characterisation as a starting point for research along
those lines. We call an \rmrs\ $\varphi'$ an \emph{extension} of the
\rmrs\ $\varphi$ if $\varphi'$ contains all the {\sc ep}s of $\varphi$ and
$D(\varphi') \supseteq D(\varphi)$.

\begin{prop}\label{thm:big-one}
  Let $\varphi, \varphi'$ be two {\sc rmrs}s.  Then $\varphi \models
  \varphi'$ iff for every solved form $S$ of $\varphi$, there is a
  solved form $S'$ of $\varphi'$ such that $S$ is an extension of
  $S'$. 
\end{prop}
\begin{proof}[Proof (sketch)]
``$\Leftarrow$'' follows from
Props.~\ref{prop:solved-forms-are-satisfiable} and
\ref{prop:models-satisfy-solved-forms}.

``$\Rightarrow$'': We construct a solved form for
$\varphi'$ by choosing a solved form for $\varphi$ and appropriate
substitutions for mapping the variables of $\varphi$ and $\varphi'$
onto each other, and removing all atoms using variables that don't
occur in $\varphi'$ .  The hard part is the proof that the result is a
solved form of $\varphi'$; this step involves proving that if $\varphi
\models \varphi'$ with the same variable assignments, then all {\sc
  ep}s in $\varphi'$ also occur in $\varphi$.
\end{proof}

%%% Local Variables: 
%%% mode: latex
%%% TeX-master: "rmrs-08"
%%% End: 
