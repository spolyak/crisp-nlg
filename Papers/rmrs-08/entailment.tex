\section{RMRS comparison as entailment}
\label{sec:entailment}

As discussed in Sections~\ref{sec:intro} and~\ref{sec:motivation}, 
{\sc rmrs}'s design aims to make it possible to compare
the semantic output of several language processors of varying depth,
and even combine them when appropriate.  The following theorem 
demonstrates that {\sc rmrs}-entailment---defined in terms of
models and or solved forms---provides the logical 
basis for such parser integration:

% NOTE: I'm fudging that this applies only to normal dominance
% constraints that are overlap-free.  We should put this back if there
% is room!
%
\begin{thm}\label{thm:big-one}
  Let $\varphi, \varphi'$ be two {\sc rmrs}s.  Then $\varphi \models
  \varphi'$ iff for every solved form $S$ of $\varphi$, there is a
  solved form $S'$ of $\varphi'$ such that $S$ is an extension of
  $S'$.
\end{thm}

This theorem doesn't provide 
efficient
algorithms for entailment checking, and of course this would be
required to make parser comparison and integration with {\sc rmrs}
practical.  Computing the solved forms of $\varphi$ and $\varphi'$ may
be exponential, and the complexity of the {\sc rmrs} validity problem
is still unknown.  However, the above theorem illustrates that our
model theory for {\sc rmrs} functions in the way it should for parser
comparison and integration, and it can thus be used to prove important
logical properties of shallow parsing systems that adopt {\sc rmrs}.

The outline proof of Theorem~\ref{thm:big-one} is as follows:
\begin{proof}
  ``$\Leftarrow$'': Let $M,\alpha$ an arbitrary model and variable
  assignment that satisfy $\varphi$.  There is a solved form $s$ of
  $\varphi$ such that $M,\alpha \models s$.  By assumption, this means
  that there is a solved form $s'$ of $\varphi'$ such that $s$ is an
  extension of $s'$.  Therefore, $M,\alpha \models s'$ and hence
  $M,\alpha \models \varphi'$.

  ``$\Rightarrow$'': Assume that $\varphi \models \varphi'$, and let
  $s$ be an arbitrary solved form of $\varphi$.  We need to show that
  we can remove atoms from $s$ such that $s$ remains an extension of
  the resulting $s'$ and $s'$ is a solved form of $\varphi'$.

  We construct $s'$ as the conjunction of all atoms in $s$ that only
  contain variables occurring in $\varphi'$ and all atom $X \dom Y$
  such that $X$ and $Y$ occur in $\varphi'$ and $(X,Y) \in D(s)$.

  It is obvious that $s$ is an extension of $s'$.  Furthermore, $s'$
  is in solved form, because $s$ was a tree already and we didn't
  introduce new edges.  The tricky part is to show that $s'$ is a
  solved form of $\varphi'$.  $s'$ contains all labeling atoms of
  $\varphi'$ according to
  Lemma~\ref{lem:entailment-preserves-labeling-atoms}; a similar
  argument holds for inequality atoms.  Finally, $D(\varphi')
  \subseteq D(s')$ because $M,\alpha \models s'$ and $s'$ is
  tree-shaped and normal and therefore contains all statements about
  dominance that are true in $M$.
\end{proof}

\begin{lemma} \label{lem:entailment-preserves-labeling-atoms}
  Under the assumptions of
  Theorem~\ref{thm:big-one}, if $\varphi \models
  \varphi'$, then $\varphi'$ only contains labeling atoms that also
  occur in $\varphi$.  
\end{lemma}
\begin{proof}
  Let $\varphi \models \varphi'$, but let $\varphi'$ contain a
  labeling atom $X \Neq f(X_1,\ldots,X_n)$ that doesn't occur in
  $\varphi$.  Call this atom $A$.

  Furthermore, let $M,\alpha \models \varphi$ arbitrary.  Because
  $M,\alpha \models \varphi'$, we must have $M(\alpha(X)) = f$.

  Now replace the label of $\alpha(X)$ in $M$ by some other symbol $g$
  of arity $n$.  Call the resulting model $M'$.  We still have
  $M',\alpha \models \varphi$ because $\varphi$ doesn't contain a
  labeling atom for $X$.  But now $M',\alpha \not\models \varphi'$.
  This contradicts the entailment assumption.
\end{proof}

It is really necessary to state
Theorem~\ref{thm:big-one} in terms of extensions of
solved forms.  It is \emph{not} true that $\varphi \models \varphi'$
implies that $\varphi$ is an extension of $\varphi'$.  A
counterexample is shown in Fig.~\ref{fig:counterexample}, where the
right-hand constraint entails the left-hand constraint (they are
equivalent), but is not an extension of it.

\begin{figure}
  \centering
  \includegraphics[scale=0.6]{pic-extension-counterexample}
  \caption{A counterexample to a simpler version of Theorem~\ref{thm:big-one}.}
  \label{fig:counterexample}
\end{figure}



%%% Local Variables: 
%%% mode: latex
%%% TeX-master: "rmrs-08"
%%% End: 
