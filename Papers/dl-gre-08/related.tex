\section{A unified perspective on GRE} \label{sec:related}

\begin{figure}
  \centering
  \begin{tabular}{l|l}
    GRE algorithm & DL variant \\ \hline
    \newcite{Dale1995} & $\mathcal{CL}$ \\
    \newcite{deemter01:_gener_refer_expres} & $\mathcal{PL}$ \\
    \newcite{dale91:_gener_refer_expres_invol_relat} & \el \\
    \newcite{Krahmer2003} & \el \\
    \newcite{kelleher06:_increm_gener_of_spatial_refer} & \el \\
    \newcite{gardent02:_gener_minim_defin_descr} & $\mathcal{ELU}_{(\neg)}$
  \end{tabular}
  \caption{DL variants used by different GRE algorithms.}
  \label{fig:related}
\end{figure}

Viewing GRE as a problem of generating DL concepts offers a new,
unified perspective of the GRE problem: It is the problem of computing
a DL concept with a certain (typically singleton) extension.  Many
existing approaches can be subsumed under this view; we have
summarized this for some of them in Fig.~\ref{fig:related}, along with
the DL fragment they use.  We already discussed some of these
approaches in Section~\ref{sec:gre}.  Furthermore, the graphs
\newcite{Krahmer2003} use to represent referring expressions can be
seen as concepts of \el\ that are only satisfied at those points at
which they can be embedded; the non-relational but negative and
disjunctive descriptions generated by
\newcite{deemter01:_gener_refer_expres} are simply concepts of
$\mathcal{PL}$; and \newcite{gardent02:_gener_minim_defin_descr} then
generalizes this into generating concepts of $\mathcal{ELU}_{(\neg)}$,
i.e.\ \el\ plus disjunction and atomic negation.

The approach presented here fits well into this landscape, and it completes the picture by showing how to generate concepts for
arbitrary bisimulation classes in \alc, which combines all connectives
used in any of these previous approaches.  Where our approach breaks
new ground is in the way these concepts are computed: It successively
refines a decomposition of the domain into subsets.  In this way, it
is reminiscent of the Incremental Algorithm, and in fact the IA can be
seen as the special case of the \el\ algorithm for the non-relational
case.

However, our approach is in contrast with
\newcite{dale91:_gener_refer_expres_invol_relat} and its successors,
such as \newcite{kelleher06:_increm_gener_of_spatial_refer}, which
attempt to generate a RE for a single individual for successive
individuals in the model and therefore run a risk of infinite regress.
Our algorithms are guaranteed to terminate (and with concepts of the
smallest relational depth that cover all the data), and infinite
regress is automatically excluded.  Perhaps closest in spirit is the
Krahmer et al.\ graph algorithm, which also computes \el\ concepts by
extending them successively; but this algorithm, too, is focused on
reference to a single individual, and it has NP-complete worst-case
complexity.







%%% Local Variables: 
%%% mode: latex
%%% TeX-master: "dl-gre-08"
%%% End: 
