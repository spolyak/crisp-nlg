\newcommand{\gM}{\mathcal{M}}
\newcommand{\gL}{\mathcal{L}}


\section{Description Languages and Bisimulation} \label{sec:bisim}

In computational linguistics, the genereting referring expression problem (the GRE
problem) can be intuitively presented as follows: given some information about
a group of objects (e.g., grounded facts about individuals in an ongoing dialogue),
and a grammar, found a grammatically correct expression (usually a noun phrase) that uniquely identify a given object.

By taking a logic perspective, we can recast the GRE problem as an inference problem.
We can think that the information about the group of objects is provided in the form
of a model $\gM$ of a given logical language $\gL$.  Now, the inference problem
associated with the GRE problem will be to find a formula in $\gL$ such that it
exacly identify a given object $o$.  Formally, if $|\cdot|^\gM$ is the interpretation
function for $\gL$ which for each formula in $\gL$ and for each model $\gM$ returns
the set of elements in $\gM$ for which the formula $\varphi$ holds, then we can define
\medskip

\noindent
{\small
\begin{tabular}{l} \hline
\textsc{$\gL$-GRE Problem}\\ \hline
\ \ \ Input: A model $\gM$, an object $o$ in the domain of $\gM$\\
\ \ \ Output: A formula $\varphi \in \gL$ such that $|\varphi|^\gM = \{o\}$\\
\hspace*{1.4cm} (if such a formula exists).\\ \hline
\end{tabular}}
\medskip

As it can be seen from the definition, the problem depends on which particular
logical language $\gL$ we are interested in, and also (as the original GRE problem)
it might terminate with failure.  Depending on the model $\gM$, the chosen object $o$
and the expressive power of $\gL$ it can be the case that there is no formula of
$\gL$ that uniquely identifies $o$.

Consider the following example.  We have three objects related as indicated in
Figure~\ref{fig-car1}.
\begin{figure}
\begin{center}
\includegraphics[scale=.8]{pic.pdf}
\end{center}
\caption{Distinguishable/indistinguishable points.}\label{fig-car1}
\end{figure}
%
The points $o$ and $o'$ can easily be distinguished in first-order logic:
the formula $R(x,x)$ is true of $o$ and false in $o'$. The corresponding
referring expression for $o$ would be ``a point relating to itself''.
But if we can only use propositional and relational information (equivalently, our grammar does not contain the word ``itself'') then $o$ and $o'$ are indistinguishable. Both
are ``points which are related to points which are related to points\ldots''

In what follows we will introduce two logical languages (called \el and \alc, see~\cite{handbook}) which can naturally be mapped to standard grammatically correct expressions.  \el allows
for conjunctions of atomic and existentially quantified relational expressions (e.g.,
``a red book on a table'') while \alc is the extension of \el with full negation
(e.g., ``a book which is not red and is not on the floor'').

\begin{definition}
Formulas of $\alc$ are generated by the following grammar:
$$
\form ::= p \mid \neg \varphi \mid \varphi \wedge \varphi' \mid \exists R. \varphi
$$
where $p$ is in the set of propositional symbols \prop, $R$ is in the set of relational symbols \rel, $\varphi$ and $\varphi'$ are in \form. $\el$ is the
negation-free fragment of $\alc$.

Formulas of both $\alc$ and $\el$ are interpreted in relational first-order models
$\gM = (\Delta^\gM,|\cdot|^\gM)$ where $W$ is an arbitrary non-empy set, and $|\cdot|^\gM$ is an interpretation
function such that:
$$
\begin{array}{ccl}
|p|^\gM & \subseteq & \Delta^\gM  \mbox{ for $p \in \prop$}\\
|R|^\gM & \subseteq & \Delta\times W  \mbox{ for $R \in \rel$}\\
|\neg \varphi|^\gM & = & W \backslash |\varphi|^\gM\\
|\varphi \wedge \varphi'|^\gM & = & |\varphi|^\gM \cap |\varphi'|^\gM\\
|\exists R.\varphi|^\gM & = & \{o \mid \mbox{for some } o' ((o,o') \in |R|^\gM)\\
& & \mbox{ and } o' \in |\varphi|^\gM \}.\\
\end{array}
$$
\end{definition}

We can now exactly characterize the problem of when a formula distinguishing two
elements in a given model exists. We will do so using the notion of \emph{bisimulaiton}.

\begin{definition}
Given a model $\gM = (\Delta^\gM,)$
\end{definition}


\cite{dovier04:_effic_algor_for_comput_bisim_equiv}

%%% Local Variables: 
%%% mode: latex
%%% TeX-master: "dl-gre-08"
%%% End: 
