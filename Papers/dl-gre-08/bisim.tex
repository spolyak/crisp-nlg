\section{Description Logics and Simulation} \label{sec:bisim}

As we mentioned, in this paper we will represent referring expressions
(REs) as formulas of description logic~\cite{baad:desc03}.  In order
to make this point, we will now first define the two description
logics we will be working with -- \alc\ and \el.


\begin{definition}
Formulas $\varphi$ of $\alc$ are generated by the following grammar:
$$
\varphi,\varphi' ::= p \mid \neg \varphi \mid \varphi \sqcap \varphi'
\mid \exists R. \varphi
$$
where $p$ is in the set of propositional symbols \prop, $R$ is in the
set of relational symbols \rel. $\el$ is the negation-free fragment of $\alc$.

Formulas of both $\alc$ and $\el$ are interpreted in ordinary
relational first-order models $\gM = (\Delta^\gM,|\cdot|^\gM)$ where
$\Delta^\gM$ is some non-empy set, and $|\cdot|^\gM$ is an
interpretation function such that:
$$
\begin{array}{ccl}
|p|^\gM & \subseteq & \Delta^\gM  \mbox{ for $p \in \prop$}\\
|R|^\gM & \subseteq & \Delta^\gM \times \Delta^\gM  \mbox{ for $R \in \rel$}\\
|\neg \varphi|^\gM & = & \Delta^\gM \backslash |\varphi|^\gM\\
|\varphi \sqcap \varphi'|^\gM & = & |\varphi|^\gM \cap |\varphi'|^\gM\\
|\exists R.\varphi|^\gM & = & \{i \mid \mbox{for some } i', (i,i') \in |R|^\gM\\
& & \mbox{ and } i' \in |\varphi|^\gM \}.\\
\end{array}
$$

Given a model $\gM$ and an individual $i$ in $\gM$, let
$\propm(i) = \{p \in \prop \mid i \in |p|^\gM\}$.
\end{definition}

Every formula of a description logic denotes a set of individuals in
the domain; thus we can use such formulas to describe sets.  For
instance, in the model in Fig.~\ref{fig:dale-haddock}b, the formula
$\mathsf{flower}$ denotes the set $\{f_1,f_2\}$; the formula
$\mathsf{flower} \sqcap \exists \mathsf{in}.\mathsf{hat}$ denotes
$\{f_2\}$; and the formula $\mathsf{flower} \sqcap \neg
\exists.\mathsf{hat}$ denotes $\{f_1\}$.

Different description logics differ in the inventory of logical
connectives they permit.  The difference between \alc\ and \el\ is
that \alc\ permits negation and \el\ doesn't.  There are many other
description logics in the literature; some that we will get back to in
Section~\ref{sec:related} are $\mathcal{CL}$ (\el\ without existential
quantification, i.e.\ only conjunctions of atoms); $\mathcal{PL}$
(\alc\ without existential quantification, i.e.\ propositional logic);
and $\mathcal{ELU}_{(\neg)}$ (\el\ plus disjunction and atomic
negation).

A key notion in studying description logics is \emph{simulation}.
We will discuss in detail the case for $\alc$; further details can be
found in~\cite{blac:moda01,kurt:expr99}.

\begin{definition}
Given a model $\gM = (\Delta^\gM,|\cdot|^\gM)$, and two individuals $i$ and $i'$ in $\gM$, an \emph{$\alc$-simulation} linking $i$ to $i'$  is a relation $Z
\subseteq \Delta^\gM \times \Delta^\gM$ such that 
\begin{itemize}
\item[i)] $(i,i') \in Z$.\\[-1.5em]
\item[ii)] if $(e_1, e_2) \in Z$ then $\propm(e_1) = \propm(e_2)$.\\[-1.5em]
\item[iii)] if $(e_1,e_2) \in Z$ and for some $R \in \rel$ there is $e_1'$ such that
$(e_1,e_1') \in |R|^\gM$ then there is $e_2'$ such that $(e_2,e_2') \in |R|^\gM$ and
$(e_1',e_2') \in Z$.\\[-1.5em]
\item[iv)] if $(e_1,e_2) \in Z$ and for some $R \in \rel$ there is $e_2'$ such that
$(e_2,e_2') \in |R|^\gM$ then there is $e_1'$ such that $(e_1,e_1') \in |R|^\gM$ and
$(e_1',e_2') \in Z$.
\end{itemize}
%A maximal subset of $\Delta^\gM$ all of whose elements are pairwise
%bisimilar is called an \emph{\alc-bisimulation class} of $\gM$.
\end{definition}

We will say that an individual $i$ is \emph{\alc-similar} to $i'$
in a given model $\gM$ if for any formula $\varphi \in \alc$ $i \in \interp{\varphi}$ then
$i' \in \interp{\varphi}$.  Notice that, given that the language includes negation,
the notion of $\alc$-similarity is symmetric: if $i$ is \alc-similar to $i'$ then
$i'$ is \alc-similar to $i$.  This is not the case for all description
logic.  In particular, the notion of $\el$-similarity is not symmetric.
Given a model $\gM$ and an individual $i$ in $\gM$, we will write \simm$_\alc(i)$, the similarity set, to all the elements of $\gM$ which are \alc-similar to $i$.x

The crucial property that makes simulation relevant for GRE is the
following theorem \cite{blac:moda01}.

\begin{theorem}\label{bisim}
  Given a finite model $\gM$ and two individuals $i$ and $i'$ in $\gM$, there is an $\alc$-simulation linking $i$ to $i'$ if and only if $i$ is $\alc$-similar to $i'$.
\end{theorem}

Notice that, because for $\alc$ the notion of similarity si symmetric, $\alc$-similar individuals are indistinguishable in the $\alc$ language.  In the example in Fig.~\ref{fig:dale-haddock}a,
the two individuals $t_1$ and $t_2$ are \alc-similar; and indeed,
they satisfy exactly the same \alc-formulas.

Although we can't go
into details here, it is possible to define a suitable notion of
$\el$-simulation for which the analogous theorem holds
\cite{kurt:expr99}.  In general, \el-simulation is a weaker notion
than \alc-simulation; for instance, in Fig.~\ref{fig:dale-haddock},
$f_1$ is \el-similar to $f_2$, but not vice-versa.

%%% Local Variables:
%%% mode: latex
%%% TeX-master: "dl-gre-08"
%%% End: 
