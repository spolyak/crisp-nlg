\newcommand{\gM}{\mathcal{M}}
\newcommand{\gL}{\mathcal{L}}


\section{Description Languages and Bisimulation} \label{sec:bisim}

One of the main ideas we want to investigate in this paper is
how we can use \emph{formulas} from description logics (see~\cite{baad:desc03}) to represent
referring expressions (RE). And how we can use the notion of \emph{bisimulation}
(see~\cite{blac:moda01}) to obtain very efficient algorithms for computing referring expressions.
 
In what follows we will introduce in detail two description logics called \el and \alc.  Intuitively, \el allows
for conjunctions of atomic and existentially quantified relational expressions (e.g.,
we can write formulas in \el that could be realized as ``a red book on a table''). \alc is the extension of \el with full negation
(e.g., we can write formulas in \alc which could be realized as ``a book which is not red and is not on the floor'').  Computing bisimulations in \alc is very efficient, but \alc can contain expressions which are difficult to realize (because of the presence of negation).  On the other hand,
\el expressions are easily realizable, but the bisimulation algorithm is more complex.

But let start by formally defining the description languages we are going to use, and the notion of bisimulation.

\begin{definition}
Formulas of $\alc$ are generated by the following grammar:
$$
\form ::= p \mid \neg \varphi \mid \varphi \sqcap \varphi' \mid \exists R. \varphi
$$
where $p$ is in the set of propositional symbols \prop, $R$ is in the set of relational symbols \rel, $\varphi$ and $\varphi'$ are in \form. $\el$ is the
negation-free fragment of $\alc$.

Formulas of both $\alc$ and $\el$ are interpreted in relational first-order models
$\gM = (\Delta^\gM,|\cdot|^\gM)$ where $W$ is an arbitrary non-empy set, and $|\cdot|^\gM$ is an interpretation
function such that:
$$
\begin{array}{ccl}
|p|^\gM & \subseteq & \Delta^\gM  \mbox{ for $p \in \prop$}\\
|R|^\gM & \subseteq & \Delta^\gM \times \Delta^\gM  \mbox{ for $R \in \rel$}\\
|\neg \varphi|^\gM & = & W \backslash |\varphi|^\gM\\
|\varphi \sqcap \varphi'|^\gM & = & |\varphi|^\gM \cap |\varphi'|^\gM\\
|\exists R.\varphi|^\gM & = & \{i \mid \mbox{for some } i' ((i,i') \in |R|^\gM)\\
& & \mbox{ and } i' \in |\varphi|^\gM \}.\\
\end{array}
$$
Given a model $\gM$ and an individual $i$ in the domain of $\gM$, let
$\propm(i) = \{p \in \prop \mid i \in |p|^\gM\}$.
\end{definition}

The previous definition formally introduces the syntax and semantics of the
logics \alc and \el.  In the paper we will sometime refer to some other description logics (mainly in Section~\ref{xxx}).  For completeness we mention them briefly here,
and we refer the reader to~\cite{baad:desc03} for further details:

\begin{center}
\begin{tabular}{lp{6cm}}
$\mathcal{CL}$ & is obtained from $\el$ by dropping existential quantification\\
$\mathcal{PL}$ & is obtained form $\alc$ by dropping existential quantification\\
$\mathcal{ELU}_{(\neg)}$ & is obtained from $\el$ by adding disjunction and atomic negation
\end{tabular}
\end{center}

\noindent
It should be clear from the definitions above that $\mathcal{CL}$ is conjunctive logic, while $\mathcal{PL}$ is propositional logic.

We now introduce the notion of \emph{bisimulation}.  We will discuss in detail
the case for the $\alc$ language.  Further details can be found
in~\cite{blac:moda01,kurt:expr99}.

\begin{definition}
Given a model $\gM = (\Delta^\gM,|\cdot|^\gM)$, two elements $i$ and $i'$ of $\Delta^\gM$ are $\alc$-bisimilar if it is possible to find a relation $Z \subseteq \Delta^\gM \times \Delta^\gM$ such that
\begin{enumerate}
\item $(i,i') \in Z$.
\item if $(e_1, e_2) \in Z$ then $\propm(e_1) = \propm(e_2)$.
\item if $(e_1,e_2) \in Z$ and for some $R \in \rel$ there is $e_1'$ such that
$(e_1,e_1') \in |R|^\gM$ then there is $e_2'$ such that $(e_2,e_2') \in |R|^\gM$ and
$(e_1',e_2') \in Z$.
\item if $(e_1,e_2) \in Z$ and for some $R \in \rel$ there is $e_2'$ such that
$(e_2,e_2') \in |R|^\gM$ then there is $e_1'$ such that $(e_1,e_1') \in |R|^\gM$ and
$(e_1',e_2') \in Z$.
\end{enumerate}
\end{definition}

Why are bisimulations relevant for computing referring expressions?
The following result provide us with a crucial characterization:

\begin{theorem}[\cite{blac:moda01}]\label{bisim}
Given a finite model $\gM$, $i$ and $i'$ are two $\alc$-bisimilar elements in $\gM$
if and only if there is no $\alc$-formula $\varphi$ such that $i \in |\varphi|^\gM$ and
$i' \not \in |\varphi|^\gM$.
\end{theorem}

In other words, $\alc$-bisimilar elements are indistinguishable in the $\alc$ language. Hence, bisimulation gives us a test for non-referenceability: if a
model $\gM$ contains two different individuals $i$ and $i'$ which are
$\alc$-bisimilar, then there is no formula of $\alc$ that can be used to uniquely
refer to $i$ or $i'$.  We will discuss which is the exact relation between
bisimulation and generation of referring expressions in the next section.
For the moment we just remark that a suitable notion of $\el$-bisimulation can also be defined~\cite{kurt:expr99}.

%%% Local Variables:
%%% mode: latex
%%% TeX-master: "dl-gre-08"
%%% End: 
