\documentclass[11pt,a4]{article}

\usepackage{url}
\usepackage{graphicx}
\usepackage{natbib}
\usepackage{amsfonts,amsthm}

\def\N{\mathbb{N}}
\newcommand{\sem}{\mathsf{sem}}
\newcommand{\self}{\mathsf{self}}
\newcommand{\produ}{\mathsf{prod}}
\newcommand{\roles}{\mathsf{roles}}
\newcommand{\Neq}{{:}}
\newcommand{\refr}{\mathsf{ref}}
\newcommand{\id}{\mathsf{id}}
\newcommand{\idsem}{\mathsf{idsem}}
\newcommand{\Vars}{\mathsf{Vars}}

\theoremstyle{plain}
\newtheorem{theorem}{Theorem}
\newtheorem{prop}[theorem]{Proposition}
\newtheorem{kor}[theorem]{Corollary}
\newtheorem{lemma}[theorem]{Lemma}
\newtheorem{verm}{Conjecture}

\theoremstyle{definition}
\newtheorem{definition}[theorem]{Definition}
\newtheorem{bem}{Remark}


\title{Sentence generation with RTGs}
\author{Alexander Koller \\ Saarland University \\
  \url{koller@mmci.uni-saarland.de}}
\date{\today\ (v3.0)}

\begin{document}

\maketitle

\section{Introduction}

\section{The generation problem of regular tree grammars}

I write $\N_n$ for the set $\{1,\ldots,n\}$.  I write $T_A(B)$ for the
set of terms $f(a_1,\ldots,a_n)$ with $f \in A$ and $a_1,\ldots,a_n
\in B$.  I also assume that the set of nodes in a tree is a set of
strings in $\N^*$ that is closed under left sister (i.e., if $ui$ is a
node of some tree and $i>1$, then $u(i-1)$ is also a node) and prefix
(i.e., if $ui$ is a node of the tree, then $u$ is too).

\begin{definition}
  A \emph{regular generation grammar (RGG)} is a tuple $G =
  (N,\Sigma,S,Pred,R,L)$ consisting of a terminal alphabet $\Sigma$, a
  nonterminal alphabet $N$, a start symbol $S \in N$, a set $Pred$ of
  \emph{predicate symbols}, a set $R$ of \emph{roles}, and a finite
  set $L$ of \emph{lexicon entries} $l = (P,\sem,r)$, where $P = A
  \rightarrow f(B_1,\ldots,B_n)$ is an RTG production rule over $N$
  and $\Sigma$, $r:\N_n \rightarrow R$ an assignment of roles to
  right-hand nonterminal occurrences that assigns the role $\self$ to
  exactly one number, and $\sem \subseteq
  T_{Pred}(\{r(1),\ldots,r(n)\})$ is the semantic representation.
\end{definition}

We write $A \rightarrow f(B_1\Neq r_1,\ldots,B_n \Neq r_n)$ to
abbreviate a production rule with a role assignment that maps $i$ to
$r_i$ for each $1 \leq i \leq n$. We write $\produ(l)$, $\roles(l)$,
and $\sem(l)$ for the production rule, the set of roles
$\{r(1),\ldots,r(n)\}$, and the $\sem$ value of the lexicon entry
respectively.

\begin{definition}
  Assume a universe $U$ and an infinite set $\Vars$ of variables. A
  \emph{derivation} of a regular generation grammar $G$ is a sequence
  $d = l_1,\ldots,l_n$ of lexicon entries from $G$ such that there are
  functions $\id:\N_n \times R \rightsquigarrow \Vars$ and
  $\refr:\Vars \rightarrow U$ such that
  \begin{enumerate}
  \item grammaticality: the sequence $\produ(l_1),\ldots,\produ(l_n)$
    is a complete RTG derivation that maps $S$ into a tree of terminal
    symbols;
  \item definedness: $\id(i,r)$ is defined iff $r \in
    \roles(l_i)$;
  \item consistent reference: if $\produ(l_k)$ expands the right-hand
    nonterminal $A:r$ in $\produ(l_i)$, then $\id(k,\self) =
    \id(j,r)$.
  \end{enumerate}

  We write $\idsem(l_i) = \{P(x_1,\ldots,x_m) \;|\;
  \mbox{$P(r_1,\ldots,r_m) \in \sem(l_i)$ and $x_k = \id(i,r_k)$ for
    all $k$}\}$, and take $||S||_M$ to be the set of all variable
  assignments $\Vars \rightarrow U$ that satisfy $S$ in all models
  over $U$ that satisfy all formulas in the set $M$.

  Then we can define a \emph{successful derivation} of $G$ for the
  communicative goal $C \subseteq T_{Pred}(U)$, the target referent $e
  \in U$, and the speaker and hearer knowledge bases $SKB, HKB
  \subseteq T_{Pred}(U)$ to be a derivation of $G$ that also satisfies
  the following conditions:

  \begin{enumerate}
  \item truthfulness: for all $i$, $\refr(\idsem(l_i)) \subseteq SKB$;
  \item communicative goal achieved: $C \subseteq \cup_{k=1}^n \refr(\idsem(l_k))$;
  \item unique reference: for all $1 \leq i \leq n$ and $r \in
    \roles(l_i)$, $|| \cup_{k=1}^n \idsem(l_k) ||_{HKB}(\id(i,r)) =
    \{\refr(\id(i,r))\}$.
  \end{enumerate}

  The \emph{sentence generation problem} of RGGs is to decide for a
  grammar $G$ and the communicative goal, target referent, and
  knowledge bases, whether there is a successful derivation.
\end{definition}



\section{RTG generation as planning}


\section{NP-completeness of the RTG generation problem}

\section{Conclusion}




\section*{Version History}

\begin{tabular}{lll}
  v 3.0 & \today & Completely revised \\
  v 2.0 & 29/10/07 & Version for Hector Geffner
\end{tabular}



\bibliographystyle{plainnat}
\bibliography{gen}

\end{document}
