\section{Introduction}
\label{sec:introduction}

\emph{Natural language generation} (NLG; \citealp{reiter00building}) is one of
the major subfields of natural language processing, concerned with computing
natural language sentences or texts that convey a given piece of information to
an audience. While the output of a generation task can take many forms,
including written text, synthesised speech, or embodied multimodal
presentations, the underlying NLG problem in each case can be modelled as a
problem of achieving a (communicative) goal by successively applying a set of
(communicative) actions. This view of NLG as goal-directed action has clear
parallels to \emph{automated planning}, which seeks to find general techniques
for efficiently solving the action sequencing problem.

Treating generation as planning has a long history in NLG, ranging from the
initial attempts of the field to utilise early planning approaches
\citep{perrault80,appelt:planning,hovy88,young94dpocl}, to the recent surge of
research \citep{Steedman-Petrick:07,KolSto07,benotti08b} seeking to capitalise
on the improvements modern planners offer in terms of efficiency and
expressiveness. This paper attempts to assess the usefulness of current planning
techniques to NLG by investigating some representative generation problems, and
evaluating whether automated planning has advanced to the point that it can
provide solutions to such NLG applications---applications that are not currently
being investigated by mainstream planning research.

The focus of this paper is twofold. First, we present two generation problems
that have recently been cast as planning problems: the sentence generation task
and the GIVE task. In the sentence generation task, we concentrate on generating
a single sentence that expresses a given meaning. In this case, a plan encodes
the necessary sentence with the actions in the plan corresponding to the
utterance of individual words \citep{KolSto07}. In the GIVE domain (``Generating
Instructions in Virtual Environments''), we describe a new shared task that was
recently posed as a challenge for the NLG community
\citep{ByrKolStrCasDalMooObe09}. GIVE uses planning as part of a larger NLG
system for generating natural-language instructions that guide a human user in
performing a given task in a virtual environment.

Second, we discuss our experiences using a set of off-the-shelf planners in
these two generation domains. Among the planners we test, we explore the
efficiency of FF \citep{HoffmannNebel01}---a planner that has arguably had the
greatest impact on recent approaches to deterministic planning---and its many
descendants, such as SGPLAN \citep{hsu06:_new_featur_in_sgplan_for}. All of the
planners we test are freely available, support an expressive subset of the
Planning Domain Definition Language (PDDL; McDermott et al.~\citeyear{PDDL}),
and have been successful on both standard planning benchmarks and the problems
of the International Planning Competition (IPC).\footnote{See
  \url{http://ipc.icaps-conference.org/} for information about recent and past
  editions of the IPC. A good overview of the deterministic track of the 2004
  instance of the IPC can be found in \citep{Hoffmann-Edelkamp:05}.}
Using these planners---and an ad-hoc Java implementation of GraphPlan
\citep{Blum1997} which serves as a baseline for certain experiments---we perform
a series of tests on a range of problem instances in our NLG domains.

Overall, our findings are mixed. On the one hand, we demonstrate that certain
planners can readily handle the \emph{search} problems that arise in our testing
domains on realistic inputs, which is promising given the challenging nature of
these tasks (e.g., the sentence generation task is NP-complete \citep{KolStr02})
On the other hand, these same planners spend tremendous amounts of time on
\emph{preprocessing} techniques that analyse the problem domain in support of
the search. (For instance, FF spends 90\% of its runtime in the sentence
generation domain on grounding out actions and literals.) As a consequence, the
off-the-shelf planners we investigated are often too slow to be useful in real
NLG applications.  Nevertheless, we offer our NLG domains and experiences as
both challenges and lessons for the planning community, in the hope that the
issues we raise might gain more attention in the future.

The remainder of this paper is structured as follows. In
Section~\ref{sec:nlg-as-planning}, we introduce the idea of NLG as planning and
briefly review the relevant literature. In Section~\ref{sec:domains}, we
describe a set of planning problems associated with two NLG tasks, namely
sentence planning and instruction generation. In Section~\ref{sec:experiments},
we report on the results of our experiments with these planning problems.
In Section~\ref{sec:discussion} we discuss our results and overall
experiences, and conclude in Section~\ref{sec:conclusion}.


%%% Local Variables: 
%%% mode: latex
%%% TeX-master: "manuscript"
%%% End: 
