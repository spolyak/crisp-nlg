\section{Conclusion}
\label{sec:conclusion}

In this paper, we investigated the usefulness of current planning technology
to natural language generation, an application area with a long tradition of
using automated planning that has recently experienced renewed interest from
NLG researchers. In particular, we evaluated the performance of several
off-the-shelf planners on a series of planning domains that arose in the
context of sentence generation and situated instruction generation.

Our results were mixed. While some of the planners we tested---in
particular, FF and SGPLAN---did an impressive job of controlling the
complexity of the search, we also found that all the planners we tested
spent too much time on preprocessing to be useful. For instance, in the
sentence generation domain, FF spent 90\% of its runtime on computing the
ground instances of the planning operators; in the instruction-giving
domain, which is very similar to Gridworld, a similar effect happened for
certain combinations of grid sizes and buttons. As things stand, we found
that this overly long preprocessing time makes current planners an
inappropriate choice for NLG applications, in any but the smallest problem
instances. Users who come to planning from outside the field, such as NLG
researchers, treat planners as black boxes. This means that search
efficiency alone is not helpful when other modules of the planner are slow.
From this perspective, we propose that the planning community should spend
some attention on optimising the preprocessing component of the problem with
similar vigour as the search itself. In particular, we propose that one line
of research might be to investigate planning algorithms that do not rely on
grounding out all operators prior to the search, but instead selectively
perform this operation when needed.

NLG and planning have a long history in common. The recent surge in
NLG-as-planning research presents valuable opportunities for both
disciplines. Clearly, NLG researchers who apply planning technology will
benefit directly from any improvements in planner effiency. Conversely, NLG
may also be a worthwhile application area for planning researchers to keep
in mind. Domains like GIVE highlight certain challenges, such as plan
execution monitoring and plan presentation (i.e., summarisation and
elaboration), but also offer a platform on which such technologies can be
evaluated in experiments with human users. Furthermore, although we have
focused on classical planning problems in this work, research related to
reasoning under uncertainty, resource management, and planning with
knowledge and sensing, can also be investigated in these settings. As such,
we believe our domains would provide interesting challenges for planners
entered in future editions of the IPC.


\section*{Acknowledgements}

This work arose in the context of the Planning and Language Interest
Group at the University of Edinburgh. The authors would like to thank
all members of this group, especially H{\'{e}}ctor Geffner and Mark
Steedman, for interesting discussions. We also thank our reviewers for
their insightful and challenging comments.  This work was supported by
the DFG Research Fellowship ``CRISP: Efficient integrated realization
and microplanning'', the DFG Cluster of Excellence ``Multimodal
Computing and Interaction'', and by the European Commission through
the PACO-PLUS project (FP6-2004-IST-4-27657).


%%% Local Variables: 
%%% mode: latex
%%% TeX-master: "manuscript"
%%% End: 
