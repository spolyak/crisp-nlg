\section{Related Work}
\label{sec:related}

Statistical methods are popular for surface realization, but have not been used in systems that integrate sentence planning. Most statistical generation approaches follow a generate-and-select strategy, first proposed by \newcite{knight1995} in their NITROGEN system. Such systems generate a set of candidate sentences using a (possibly overgenerating) grammar and then select the best output sentence by applying a statistical language model. This family includes systems such as HALogen \cite{langkildeknight1998,langkilde2000} and OpenCCG \cite{whitebaldridge2003}.  The FERGUS system \cite{bangalorerambow2000} is a variant of this approach which, like PCRISP, employs TAG. It first assigns elementary trees to each entity in the input sentence plan using a statistical tree model and then computes the most likely derivation using only these trees with an n-gram model on the output sentence. An alternative to the n-gram based generate and select approach is to use a probabilistic grammar model, like PTAG, trained on automatic parses \cite{zhongstent2005}. A related approach uses a model over local decisions of the generation system itself \cite{belz2008}. Both models can either be used to discriminate a set of output candidates, or more directly to choose the next best decision locally. Our approach is similar in that it uses PTAG to find the most likely output structure. However, the previous work discussed so far addresses surface realization only. We extend this to a statistical NLG algorithm which does surface realization and sentence planning at the same time.

Our treatment of integrated sentence planning and surface realization as planning is inherited directly from CRISP \cite{kollerstone2007}.  Planning has long played a role in generation, but has focused on discourse planning instead of specifically addressing sentence generation \cite{hovy1988,appelt1992}. The applicability of these ideas was limited at that time because efficient planning technology was not available. Recently the development of more efficient planning algorithms \cite{hoffmannnebel2001} spawned a renewed interest in planning for NLG. CRISP uses such algorithms to efficiently solve the sentence generation problem defined by SPUD \cite{Stone2003a}. SPUD, which instead uses an incomplete greedy algorithm, is based on a TAG  whose trees are augmented with semantic and pragmatic constraints. Given a communicative goal, a solution to the SPUD problem realizes this goal and simultaneously selects referring expressions. The next section explores CRISP in more detail.



%%% Local Variables:  
%%% mode: latex 
%%% TeX-master: "pcrisp-10" 
%%% End: 

