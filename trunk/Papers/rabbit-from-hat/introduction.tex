\section{Introduction}
\label{sec:introduction}

In the standard architecture of a natural language generation (NLG)
system, the problem of sentence generation is split into two separate
steps -- sentence planning and surface realization
\cite{reiter00building}.  However, it has been shown that by coupling
sentence planning and realization more closely, an NLG system such as
SPUD \cite{Stone2003a} can sometimes generate more succinct
expressions than a pipelined system could \cite{Stone1998a}.  In order
to express the instruction to remove the indicated rabbit from the
indicated hat in Fig.~\ref{fig:rabbits}, a pipelined system would have
to generate the sentence 

\enumsentence{Take the rabbit in the hat from the hat
that contains a rabbit.\label{ex:rabbit1}}

  A system that can exploit the fact that the
two noun phrases are connected via the verb (i.e., which makes
realization information available while referring expressions are
generated) could instead generate the more succinct

\enumsentence{Take the rabbit from the hat. \label{ex:rabbit2}}

\newcite{KolSto07} presented an implementation of the core of SPUD as
a planning problem, and demonstrated that SPUD-style integrated
sentence planning and realization can be performed efficiently by
applying modern planning technology.  However, their approach is
incapable of generating (\ref{ex:rabbit2}) because it requires the
first NP to distinguish the rabbit from all other things that are
inside other things, and likewise for the second NP.

In this paper, we will show how to extend their approach to sentence
generation as planning in such a way that it is capable of generating
referring expressions that are as succinct as SPUD's.  The relevance
of this change goes beyond isolated examples such as
(\ref{ex:rabbit1}): It requires us to model in what way adding
information to one referring expression constraints the distractors of
another.  In a first step, we will show how the Koller and Stone model
can be directly extended to take different divisions of labor between
the two referring expressions into account.  However, we will then
show that this leads to unacceptable runtimes.  The reason for this
slowdown is that the initial state and the universe considered by the
planner grow, and that the planner must make decisions about how to
distribute distractors over the different REs top-down, rather than
letting the different REs constrain each other.

In a second step, we will then present a novel approach to generation
as planning in which the planning system (in our case, PKS
\cite{Petrick-Bacchus:02}) is allowed to access external reasoning
modules.  We now keep track of a constraint network representing the
distractors of all REs \cite{dale91:_gener_refer_expres_invol_relat}
in a separate module; the planner can add new information to this
module and query it.  This allows us to keep the planning state
manageable and delay the decision about the distractors for each RE
until these REs are actually being built.  We evaluate the runtime of
our system \todo{blah}.  Attaching external reasoning modules to a
planner is a novel technique in planning, and far more general than
just keeping track of accessing a constraint network; this increase in
expressive power will be very useful for other applications of
planning-based generation systems in the future.

\todo{need to explain how distractor sets for the REs are created}


The paper is structured as follows. We will first briefly review
planning (Section~\ref{sec:planning}) and Koller and Stone's
generation as planning approach (Section~\ref{sec:gen-as-planning}).
Then we will discuss the examples (\ref{ex:rabbit1}) and
(\ref{ex:rabbit2}) in more detail and present the first, failed
attempt to computing (\ref{ex:rabbit1}) by planning, in
Section~\ref{sec:problem}.  Section~\ref{sec:howto} then describes how
we get a more efficient encoding of the generation problem by
attaching external modules to the planner.  Finally,
Section~\ref{sec:evaluation} evaluates the performance of our system,
and Section~\ref{sec:conclusion} concludes.





%%% Local Variables: 
%%% mode: latex
%%% TeX-master: "rabbit-from-hat"
%%% End: 
