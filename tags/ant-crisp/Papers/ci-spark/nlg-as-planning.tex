\section{NLG as Planning} \label{sec:nlg-as-planning}

The task of generating natural language from semantic representations (NLG) is
typically split into two parts: the \emph{discourse planning} task, which
selects the information to be conveyed and structures it into sentence-sized
chunks, and the \emph{sentence generation} task, which then translates each of
these chunks into natural language sentences. The sentence generation task is
often divided into two parts of its own---the \emph{sentence planning} task,
which enriches the input by, e.g., determining object references and selecting
some lexical material, and the \emph{surface realization} task, which maps the
enriched meaning representation into a sentence using a grammar. The chain of
domain planning, sentence planning, and surface realization is sometimes called
the ``NLG pipeline'' \citep{reiter00building}.

Viewing generation as a planning problem has a long tradition in the NLG
literature. \citet{perrault80} presented an approach to discourse planning
in which the planning operators represented individual speech acts such as
``request'' and ``inform''. This idea was later expanded, e.g., by
\citet{young94dpocl}. On the other hand, researchers such as
\citet{appelt:planning} and \citet{hovy88} used techniques from
hierarchical planning to expand a high-level plan consisting of speech acts
into more detailed specifications of individual sentences. Although these
systems covered some aspects of sentence planning, they also used very
expressive logics designed to reason about beliefs and intentions, in order
to represent the planning state and the planning operators. Most of these
systems also used ad-hoc planning algorithms with rather naive search
strategies, which did not scale well to realistic inputs. As a consequence,
the NLG-as-planning approach was mostly marginalized throughout the 1990s.

More recently, there has been a string of publications by various authors
with a renewed interest in the generation-as-planning approach, motivated
by the ongoing development of increasingly more efficient and expressive
planners. For instance, \cite{KolSto07} propose an approach to sentence
generation (i.e., the sentence planning and surface realization modules of
the pipeline) as planning---an approach we explore in more detail below
(Section~\ref{sec:domain-crisp}). \citet{Steedman-Petrick:07} revisit the
analysis of indirect speech acts with modern planning technology, viewing
the problem as an instance of planning with incomplete information and
sensing actions. In addition, \citet{benotti08b} uses planning to explain
the accommodation of presuppositions, and
\citet{brenner08:_contin_multiag_plann_approac_to_situat_dialog} use
multi-agent planning to model the joint problem solving behaviour of agents
in a situated dialogue. While these approaches focus on different issues
compared to the 1980's NLG-as-planning literature, they all apply
\emph{existing}, well-understood planning approaches to linguistic
problems, in order to utilise the rich set of modelling tools provided by
modern planners, and in the hope that such planners can efficiently solve
the hard search problems that arise in NLG \citep{KolStr02}. This paper
aims to investigate whether existing planners achieve this latter goal.


%%% Local Variables: 
%%% mode: latex
%%% TeX-master: "manuscript"
%%% End: 
