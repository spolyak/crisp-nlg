\section{Succinct expressions in planning}
\label{sec:howto}

The primary reason for FF's failure to compute a plan is due to the
fact that it computes all instances of all actions and literals before
it starts the search.  This heuristic allows it to avoid performing
unifications during the search, and is thus a good idea on the
planning competition benchmarks, whose domain sizes are not very
large.  However, in our case FF has to deal with potentially huge
numbers of instances.  If we assume a maximum plan size of 10, the
action \textbf{S-takefrom-1} alone has $31 \cdot 15^3 \cdot 4 =
418,500$ instances -- and there are ten copies of it!  For grammars of
more realistic sizes, it is easy to imagine how the number of
instances might run into the trillions.

A second problem is that FF, which is a forward planner, must choose
an instance of \textbf{S-takefrom-1} before it generates the referring
expressions that will be realized in the two noun phrases.  However,
at this point the planner doesn't know yet what choices of distractor
sets will be convenient to realize in the REs.  In the above example,
the only choice that can be realized easily is $s_3$ -- in the three
other cases, there is no single noun that will eliminate all
distractors --, but the planner must still try all three
instances. \todo{is this true? or can ff do something clever with the
  heuristic?}  In a sense, we have replaced the problems that arise
from a pipeline model which sequences sentence planning before
realization with the converse problems of performing realization
before the sentence planner had a chance to suggest REs.

Finally, a significant amount of runtime is spent on actually
computing the different splits before the planner itself is run.  In
the worst case, a domain with $n$ individuals may require us to
distinguish \todo{what?} different splits.  These not only blow up the
size of the initial state, but they also take \todo{a long} time to
compute.  In our implementation, computing the initial state for four
rabbits, hats, flowers, and bathtubs each takes about 40 minutes.
This problem is due to the fact that we don't know for what
individuals and relations we will need to know the splits during the
planning process, and so we must compute all of them to be on the safe
side.

We will now solve both of these problems by switching from FF to the
PKS planner \cite{Petrick-Bacchus:02} and extending PKS in two
different ways: allowing terms including functions in the effects, and
attaching external reasoning modules to the planner.  PKS computes
instances of actions and literals only by need, when they are actually
used by the planner, and thus automatically avoids the first problem
mentioned above.  Furthermore, by moving all the bookkeeping about
distractors into an external module which maintains a constraint
network, we avoid the second and third problem as well.  The search
for a realization is still driven by the planner; the planner then
queries the constraint network as an external module.

\todo{now we have to talk about how to actually do it}



%%% Local Variables: 
%%% mode: latex
%%% TeX-master: "rabbit-from-hat"
%%% End: 
